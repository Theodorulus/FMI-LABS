% Options for packages loaded elsewhere
\PassOptionsToPackage{unicode}{hyperref}
\PassOptionsToPackage{hyphens}{url}
%
\documentclass[
]{article}
\usepackage{lmodern}
\usepackage{amssymb,amsmath}
\usepackage{ifxetex,ifluatex}
\ifnum 0\ifxetex 1\fi\ifluatex 1\fi=0 % if pdftex
  \usepackage[T1]{fontenc}
  \usepackage[utf8]{inputenc}
  \usepackage{textcomp} % provide euro and other symbols
\else % if luatex or xetex
  \usepackage{unicode-math}
  \defaultfontfeatures{Scale=MatchLowercase}
  \defaultfontfeatures[\rmfamily]{Ligatures=TeX,Scale=1}
\fi
% Use upquote if available, for straight quotes in verbatim environments
\IfFileExists{upquote.sty}{\usepackage{upquote}}{}
\IfFileExists{microtype.sty}{% use microtype if available
  \usepackage[]{microtype}
  \UseMicrotypeSet[protrusion]{basicmath} % disable protrusion for tt fonts
}{}
\makeatletter
\@ifundefined{KOMAClassName}{% if non-KOMA class
  \IfFileExists{parskip.sty}{%
    \usepackage{parskip}
  }{% else
    \setlength{\parindent}{0pt}
    \setlength{\parskip}{6pt plus 2pt minus 1pt}}
}{% if KOMA class
  \KOMAoptions{parskip=half}}
\makeatother
\usepackage{xcolor}
\IfFileExists{xurl.sty}{\usepackage{xurl}}{} % add URL line breaks if available
\IfFileExists{bookmark.sty}{\usepackage{bookmark}}{\usepackage{hyperref}}
\hypersetup{
  pdftitle={R Notebook},
  hidelinks,
  pdfcreator={LaTeX via pandoc}}
\urlstyle{same} % disable monospaced font for URLs
\usepackage[margin=1in]{geometry}
\usepackage{color}
\usepackage{fancyvrb}
\newcommand{\VerbBar}{|}
\newcommand{\VERB}{\Verb[commandchars=\\\{\}]}
\DefineVerbatimEnvironment{Highlighting}{Verbatim}{commandchars=\\\{\}}
% Add ',fontsize=\small' for more characters per line
\usepackage{framed}
\definecolor{shadecolor}{RGB}{248,248,248}
\newenvironment{Shaded}{\begin{snugshade}}{\end{snugshade}}
\newcommand{\AlertTok}[1]{\textcolor[rgb]{0.94,0.16,0.16}{#1}}
\newcommand{\AnnotationTok}[1]{\textcolor[rgb]{0.56,0.35,0.01}{\textbf{\textit{#1}}}}
\newcommand{\AttributeTok}[1]{\textcolor[rgb]{0.77,0.63,0.00}{#1}}
\newcommand{\BaseNTok}[1]{\textcolor[rgb]{0.00,0.00,0.81}{#1}}
\newcommand{\BuiltInTok}[1]{#1}
\newcommand{\CharTok}[1]{\textcolor[rgb]{0.31,0.60,0.02}{#1}}
\newcommand{\CommentTok}[1]{\textcolor[rgb]{0.56,0.35,0.01}{\textit{#1}}}
\newcommand{\CommentVarTok}[1]{\textcolor[rgb]{0.56,0.35,0.01}{\textbf{\textit{#1}}}}
\newcommand{\ConstantTok}[1]{\textcolor[rgb]{0.00,0.00,0.00}{#1}}
\newcommand{\ControlFlowTok}[1]{\textcolor[rgb]{0.13,0.29,0.53}{\textbf{#1}}}
\newcommand{\DataTypeTok}[1]{\textcolor[rgb]{0.13,0.29,0.53}{#1}}
\newcommand{\DecValTok}[1]{\textcolor[rgb]{0.00,0.00,0.81}{#1}}
\newcommand{\DocumentationTok}[1]{\textcolor[rgb]{0.56,0.35,0.01}{\textbf{\textit{#1}}}}
\newcommand{\ErrorTok}[1]{\textcolor[rgb]{0.64,0.00,0.00}{\textbf{#1}}}
\newcommand{\ExtensionTok}[1]{#1}
\newcommand{\FloatTok}[1]{\textcolor[rgb]{0.00,0.00,0.81}{#1}}
\newcommand{\FunctionTok}[1]{\textcolor[rgb]{0.00,0.00,0.00}{#1}}
\newcommand{\ImportTok}[1]{#1}
\newcommand{\InformationTok}[1]{\textcolor[rgb]{0.56,0.35,0.01}{\textbf{\textit{#1}}}}
\newcommand{\KeywordTok}[1]{\textcolor[rgb]{0.13,0.29,0.53}{\textbf{#1}}}
\newcommand{\NormalTok}[1]{#1}
\newcommand{\OperatorTok}[1]{\textcolor[rgb]{0.81,0.36,0.00}{\textbf{#1}}}
\newcommand{\OtherTok}[1]{\textcolor[rgb]{0.56,0.35,0.01}{#1}}
\newcommand{\PreprocessorTok}[1]{\textcolor[rgb]{0.56,0.35,0.01}{\textit{#1}}}
\newcommand{\RegionMarkerTok}[1]{#1}
\newcommand{\SpecialCharTok}[1]{\textcolor[rgb]{0.00,0.00,0.00}{#1}}
\newcommand{\SpecialStringTok}[1]{\textcolor[rgb]{0.31,0.60,0.02}{#1}}
\newcommand{\StringTok}[1]{\textcolor[rgb]{0.31,0.60,0.02}{#1}}
\newcommand{\VariableTok}[1]{\textcolor[rgb]{0.00,0.00,0.00}{#1}}
\newcommand{\VerbatimStringTok}[1]{\textcolor[rgb]{0.31,0.60,0.02}{#1}}
\newcommand{\WarningTok}[1]{\textcolor[rgb]{0.56,0.35,0.01}{\textbf{\textit{#1}}}}
\usepackage{graphicx,grffile}
\makeatletter
\def\maxwidth{\ifdim\Gin@nat@width>\linewidth\linewidth\else\Gin@nat@width\fi}
\def\maxheight{\ifdim\Gin@nat@height>\textheight\textheight\else\Gin@nat@height\fi}
\makeatother
% Scale images if necessary, so that they will not overflow the page
% margins by default, and it is still possible to overwrite the defaults
% using explicit options in \includegraphics[width, height, ...]{}
\setkeys{Gin}{width=\maxwidth,height=\maxheight,keepaspectratio}
% Set default figure placement to htbp
\makeatletter
\def\fps@figure{htbp}
\makeatother
\setlength{\emergencystretch}{3em} % prevent overfull lines
\providecommand{\tightlist}{%
  \setlength{\itemsep}{0pt}\setlength{\parskip}{0pt}}
\setcounter{secnumdepth}{-\maxdimen} % remove section numbering

\title{R Notebook}
\author{}
\date{\vspace{-2.5em}}

\begin{document}
\maketitle

\#Exercitiul 1

O familie are doi copii. Care este probabilitatea ca ambii copii sa fie
baieti stiind ca cel putin unul dintre copii este baiat? Care este
probabilitate ca ambii copii sa fie baieti stiind ca cel mai tanar este
baiat?

Consideram evenimentele \(A = \{\text{Familia are ambii copii B}\}\) si
\(B = \{\text{Familia are cel putin un B}\}\) si vrem sa determinam
probabilitarea

\[
\mathbb{P}(A|B) = \frac{\mathbb{P}(A\cap B)}{\mathbb{P}(B)} = \frac{\mathbb{P}(A)}{\mathbb{P}(B)}\approx \frac{N(A)}{N(B)}\
\]

\begin{Shaded}
\begin{Highlighting}[]
\CommentTok{# Ctrl + Alt + I - generam chunk R}
\CommentTok{# Ctrl + Shift + Enter - rulam codul din chunk-ul R}

\NormalTok{N =}\StringTok{ }\DecValTok{10} \OperatorTok{^}\StringTok{ }\DecValTok{5}

\NormalTok{scopil1 =}\StringTok{ }\KeywordTok{sample}\NormalTok{(}\KeywordTok{c}\NormalTok{(}\StringTok{"F"}\NormalTok{, }\StringTok{"B"}\NormalTok{), N, }\DataTypeTok{replace =} \OtherTok{TRUE}\NormalTok{)}
\NormalTok{scopil2 =}\StringTok{ }\KeywordTok{sample}\NormalTok{(}\KeywordTok{c}\NormalTok{(}\StringTok{"F"}\NormalTok{, }\StringTok{"B"}\NormalTok{), N, }\DataTypeTok{replace =} \OtherTok{TRUE}\NormalTok{)}

\NormalTok{nr_fam_2B =}\StringTok{ }\KeywordTok{sum}\NormalTok{(scopil1 }\OperatorTok{==}\StringTok{ "B"} \OperatorTok{&}\StringTok{ }\NormalTok{scopil2 }\OperatorTok{==}\StringTok{ "B"}\NormalTok{)}
\NormalTok{nr_fam_cp1B =}\StringTok{ }\KeywordTok{sum}\NormalTok{(scopil1 }\OperatorTok{==}\StringTok{ "B"} \OperatorTok{|}\StringTok{ }\NormalTok{scopil2 }\OperatorTok{==}\StringTok{ "B"}\NormalTok{)}

\NormalTok{nr_fam_pB =}\StringTok{ }\KeywordTok{sum}\NormalTok{(scopil1 }\OperatorTok{==}\StringTok{ "B"}\NormalTok{)}


\NormalTok{p1 =}\StringTok{ }\NormalTok{nr_fam_2B }\OperatorTok{/}\StringTok{ }\NormalTok{nr_fam_cp1B}
\NormalTok{p_fam_2B =}\StringTok{ }\NormalTok{nr_fam_2B }\OperatorTok{/}\StringTok{ }\NormalTok{N}
\NormalTok{p1}
\end{Highlighting}
\end{Shaded}

\begin{verbatim}
## [1] 0.3322125
\end{verbatim}

\begin{Shaded}
\begin{Highlighting}[]
\NormalTok{p2 =}\StringTok{ }\NormalTok{nr_fam_2B }\OperatorTok{/}\StringTok{ }\NormalTok{nr_fam_pB}
\NormalTok{p2}
\end{Highlighting}
\end{Shaded}

\begin{verbatim}
## [1] 0.4978206
\end{verbatim}

\#Exercitiul 2

Problema Monty Hall

\begin{Shaded}
\begin{Highlighting}[]
\NormalTok{Monty_Hall =}\StringTok{ }\ControlFlowTok{function}\NormalTok{(}\DataTypeTok{usa_masina =} \DecValTok{1}\NormalTok{, }\DataTypeTok{usa_aleasa =} \DecValTok{2}\NormalTok{, }\DataTypeTok{strategia =} \DecValTok{1}\NormalTok{)\{}
  
\NormalTok{  usi =}\StringTok{ }\DecValTok{1}\OperatorTok{:}\DecValTok{3}
  
  \CommentTok{#Monty Hall intoarce usa necastigatoare}
  
  \ControlFlowTok{if}\NormalTok{(usa_aleasa }\OperatorTok{!=}\StringTok{ }\NormalTok{usa_masina)\{}
\NormalTok{    usa_monty =}\StringTok{ }\NormalTok{usi[}\OperatorTok{-}\KeywordTok{c}\NormalTok{(usa_aleasa, usa_masina)]}
\NormalTok{  \}}
  \ControlFlowTok{else}\NormalTok{\{}
\NormalTok{    usa_monty =}\StringTok{ }\KeywordTok{sample}\NormalTok{(usi[}\OperatorTok{-}\NormalTok{usa_aleasa], }\DecValTok{1}\NormalTok{)}
\NormalTok{  \}}
  
  \CommentTok{#print(paste0("Monty Hall intoarce usa cu numarul ", usa_monty, ", in spatele careia se afla o capra."))}
  
  \ControlFlowTok{if}\NormalTok{(strategia }\OperatorTok{==}\DecValTok{2}\NormalTok{)\{}
\NormalTok{    usa_aleasa =}\StringTok{ }\NormalTok{usi[}\OperatorTok{-}\KeywordTok{c}\NormalTok{(usa_aleasa, usa_monty)]}
\NormalTok{  \}}
  
\NormalTok{  raspuns =}\StringTok{ }\NormalTok{usa_masina }\OperatorTok{==}\NormalTok{usa_aleasa}
  
  \CommentTok{#if(raspuns)\{}
  \CommentTok{#  print("Ai castigat!")}
  \CommentTok{#\}}
  \CommentTok{#else\{}
  \CommentTok{#  print(paste0("Ai pierdut! Masina se afla in spatele usii cu numarul ", usa_masina, }
  \CommentTok{#        ", dar tu ai ales usa cu numarul ", usa_aleasa))}
  \CommentTok{#\}}
  \KeywordTok{return}\NormalTok{(raspuns)}
\NormalTok{\}}
\end{Highlighting}
\end{Shaded}

\begin{Shaded}
\begin{Highlighting}[]
\NormalTok{joc1 =}\StringTok{ }\KeywordTok{Monty_Hall}\NormalTok{(}\DecValTok{2}\NormalTok{, }\DecValTok{2}\NormalTok{, }\DecValTok{1}\NormalTok{)}
\NormalTok{joc2 =}\StringTok{ }\KeywordTok{Monty_Hall}\NormalTok{(}\DecValTok{2}\NormalTok{, }\DecValTok{2}\NormalTok{, }\DecValTok{2}\NormalTok{)}
\end{Highlighting}
\end{Shaded}

\begin{Shaded}
\begin{Highlighting}[]
\NormalTok{N =}\StringTok{ }\DecValTok{10}\OperatorTok{^}\DecValTok{5}

\KeywordTok{set.seed}\NormalTok{(}\DecValTok{123}\NormalTok{)}

\KeywordTok{sample}\NormalTok{(}\DecValTok{1}\OperatorTok{:}\DecValTok{3}\NormalTok{, }\DecValTok{10}\NormalTok{, }\DataTypeTok{replace =}\NormalTok{ T)}
\end{Highlighting}
\end{Shaded}

\begin{verbatim}
##  [1] 3 3 3 2 3 2 2 2 3 1
\end{verbatim}

\begin{Shaded}
\begin{Highlighting}[]
\NormalTok{usa_m =}\StringTok{ }\KeywordTok{sample}\NormalTok{(}\DecValTok{1}\OperatorTok{:}\DecValTok{3}\NormalTok{, N, }\DataTypeTok{replace =} \OtherTok{TRUE}\NormalTok{)}
\NormalTok{usa_a =}\StringTok{ }\DecValTok{2}

\NormalTok{strateg =}\StringTok{ }\DecValTok{2}

\NormalTok{start =}\StringTok{ }\KeywordTok{proc.time}\NormalTok{()}

\NormalTok{rezultate_joc =}\StringTok{ }\KeywordTok{logical}\NormalTok{(N)}

\ControlFlowTok{for}\NormalTok{(i }\ControlFlowTok{in} \DecValTok{1}\OperatorTok{:}\NormalTok{N)\{}
\NormalTok{  rezultate_joc[i] =}\StringTok{ }\KeywordTok{Monty_Hall}\NormalTok{(usa_m[i], usa_a, strateg)}
\NormalTok{\}}

\NormalTok{prob1 =}\StringTok{ }\KeywordTok{sum}\NormalTok{(rezultate_joc) }\OperatorTok{/}\StringTok{ }\NormalTok{N}
\NormalTok{prob1}
\end{Highlighting}
\end{Shaded}

\begin{verbatim}
## [1] 0.66698
\end{verbatim}

\begin{Shaded}
\begin{Highlighting}[]
\KeywordTok{proc.time}\NormalTok{() }\OperatorTok{-}\StringTok{ }\NormalTok{start}
\end{Highlighting}
\end{Shaded}

\begin{verbatim}
##    user  system elapsed 
##    0.44    0.00    0.44
\end{verbatim}

\begin{Shaded}
\begin{Highlighting}[]
\NormalTok{Monty_vect =}\StringTok{ }\KeywordTok{Vectorize}\NormalTok{(Monty_Hall, }\StringTok{"usa_masina"}\NormalTok{)}

\NormalTok{start2 =}\StringTok{ }\KeywordTok{proc.time}\NormalTok{()}

\NormalTok{prob_s1 =}\StringTok{ }\KeywordTok{sum}\NormalTok{(}\KeywordTok{Monty_vect}\NormalTok{(usa_m, usa_a, strateg)) }\OperatorTok{/}\StringTok{ }\NormalTok{N}
\NormalTok{prob_s1}
\end{Highlighting}
\end{Shaded}

\begin{verbatim}
## [1] 0.66698
\end{verbatim}

\begin{Shaded}
\begin{Highlighting}[]
\KeywordTok{proc.time}\NormalTok{() }\OperatorTok{-}\StringTok{ }\NormalTok{start2}
\end{Highlighting}
\end{Shaded}

\begin{verbatim}
##    user  system elapsed 
##    0.45    0.00    0.45
\end{verbatim}

\end{document}
